\section{Countermeasures}

\begin{frame}[fragile]{Countermeasures i}
    \begin{itemize}
        \item The cleaning of LightNeuron is not an easy task. \textbf{Simply removing the two malicious files will break
Microsoft Exchange}, preventing everybody in the organization from sending and receiving emails.
        \item Before actually removing the files, the malicious Transport Agent should be disabled
        \item open \texttt{<ExchangeInstallFolder>/TransportRoles/Agents/agents.config} and check every DLL signature.
        \item disable the malicious Transport Agents, and after that it is possible to safely remove the infected files.
    \end{itemize}
\end{frame}

\begin{frame}[fragile]{Countermeasures ii}
Given that attackers have gained administrative privileges on the Exchange server, there are no bulletproof
        mitigations against this threat; however there are some recommendations worth to mention:
    \begin{itemize}
        \item Use dedicated accounts for the administration of Exchange servers with strong, unique passwords
        and, if possible, 2FA.
        \item Monitor closely the usage of these accounts (DLL uses a lot of log files)
        \item Restrict PowerShell execution
        \item Regulary check all the installed Transport Agents
    \end{itemize}
\end{frame}

\begin{frame}[fragile]{Microsoft’s View}
    \emph{``an attacker would need to have administrative access on an Exchange server as a member of the Exchange
     Administrator group in an organization’s Active Directory.  Exchange Administrator accounts are tightly controlled,
      and membership cannot be obtained by persuading a victim to click through a security warning or elevation request.''} \cite{Petri}
\end{frame}
